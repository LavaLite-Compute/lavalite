\documentclass[11pt,a4paper]{article}
\usepackage[utf8]{inputenc}
\usepackage[T1]{fontenc}
\usepackage{lmodern}
\usepackage{verbatim}
\usepackage{hyperref}

\title{\textbf{LavaLite Administration Guide}}
\author{\textit{LavaLite Project}}
\date{\today}

\begin{document}

\maketitle
\tableofcontents

\section{\textbf{Overview}}
LavaLite is a lightweight batch system derived from Platform Lava 1.0.
This guide describes how to manage its daemons, verify system health, and understand its installation layout.

\section{\textbf{Installation Layout}}
LavaLite installs into a self-contained directory under \texttt{/opt/lavalite-\textit{major.minor.bug}}. The layout separates binaries, configuration, logs, and runtime data.

\subsection{\textbf{Directory Tree}}
\begin{verbatim}
lavalite-0.1/
|- bin/
|- etc/
|- include/
|- lib/
|- log/
|- sbin/
|- share/
|  |- man/
|     |- man1/
|     |- man5/
|     |- man8/
|- work/
   |- info/
\end{verbatim}

\section{\textbf{Configuration}}
LavaLite currently retains the flat LSF-style configuration structure. All config files reside in the \texttt{etc/} directory.

\begin{verbatim}
etc/
|- lsb.hosts
|- lsb.params
|- lsb.queues
|- lsb.users
|- lsf.cluster.lavalite
|- lsf.conf
|- lsf.shared
\end{verbatim}

\section{\textbf{Systemd Units}}
Users must install the following unit files manually:

\begin{verbatim}
lavalite-lim.service       # load information manager
lavalite-sbatchd.service   # per-node batch daemon
lavalite-mbatchd.service   # master scheduler
\end{verbatim}

\textbf{Important:} \textbf{sbatchd is the parent of mbatchd. You must stop sbatchd before stopping mbatchd, or before running mbatchd in debug mode.}

\section{\textbf{llctl Command}}
The \textbf{llctl} command is the unified control interface for LavaLite. It wraps systemd operations and provides internal daemon control.

\subsection{\textbf{Daemon Lifecycle}}
\begin{verbatim}
llctl lim start|stop
llctl sbatchd start|stop
llctl mbatchd start|stop
\end{verbatim}

\subsection{\textbf{Internal Operations}}
\begin{verbatim}
llctl lim lock|unlock              # clears on reboot
llctl mbatchd queue open|close Q1
llctl mbatchd host open|close nodeA
\end{verbatim}

\subsection{\textbf{Status (Priority 2)}}
\begin{verbatim}
llctl status                       # shows all daemon states
\end{verbatim}

\subsection{\textbf{Examples}}
\begin{verbatim}
llctl sbatchd stop
llctl mbatchd start
llctl lim lock
llctl mbatchd queue close Q1
llctl mbatchd host open nodeA
llctl status
\end{verbatim}

\section{\textbf{Health Checks}}
Each daemon supports a basic health probe.

\subsection{\textbf{mbatchd}}
\begin{itemize}
  \item Connect to \texttt{/run/lavalite/mbatchd.sock}
  \item Send \texttt{PING}
  \item Expect \texttt{PONG <version>}
\end{itemize}

\subsection{\textbf{sbatchd and lim}}
\begin{itemize}
  \item Ping their local sockets
  \item Verify heartbeat timestamps are current
\end{itemize}

\section{\textbf{Security}}
Use SSH keys for remote operations. Optional sudo rules can allow controlled access to systemd units.

\section{\textbf{Troubleshooting}}
\begin{itemize}
  \item Run \texttt{sync} if virtiofs caching causes delays.
  \item Use \texttt{journalctl -u lavalite-mbatchd} for logs.
  \item Verify SSH access between nodes if \texttt{llctl} times out.
\end{itemize}

\end{document}

